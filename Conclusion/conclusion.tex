\section{Section D}
\subsection{Compare and discuss $\delta _{AN}$ and $\delta _{FE}$}
Comparing two maximum deflections, $\delta _{AN}$ and $\delta _{FE}$ can find that:the experiments' result  is bigger than the result from FEA. The difference rate for Mild Steel is approximately $16.55\%$, and for Aluminum is about 12.18$\%$. Therefore, the difference rate of Aluminium is relatively smaller, which may be due to the bigger modulus of elasticity of Mild Steel. These data can be obtained from Table \ref{T 4.1} and \ref{T 4.2}.

\begin{minipage}[c]{0.5\textwidth}
	\captionof{table}{Difference of Mild Steel}
	\label{T 4.1}
	\centering
	\begin{tabular}{@{}ccc@{}}
		\toprule
		Loading & Difference & Difference rate \\ \midrule
		50N     & 0.01906 mm & 16.5681$\%$     \\
		100N    & 0.03803 mm & 16.5298$\%$     \\
		150N    & 0.05709 mm & 16.5426$\%$     \\ \midrule
		Average &            & 16.55$\%$           \\ \bottomrule
	\end{tabular}
\end{minipage}
\begin{minipage}[c]{0.5\textwidth}
	\captionof{table}{Difference of Alminium}
	\label{T 4.2}
	\centering
		\begin{tabular}{@{}ccc@{}}
			\toprule
			Loading & Difference & Difference rate \\ \midrule
			50N     & 0.03944 mm & 12.1856$\%$     \\
			100N    & 0.07887 mm & 12.1839$\%$     \\
			150N    & 0.11831 mm & 12.1845$\%$     \\ \midrule
			Average &            & 12.18$\%$           \\ \bottomrule
		\end{tabular}
\end{minipage}
\newpage
\subsection{Error analysis and optimisation}
There are a number of possible reasons for the errors in Part I of Section D.
\begin{enumerate}
	\item Metallic materials are affected by the shape memory effect. The specimen used in the experiment has been used repeatedly, and the shape has changed. In finite element analysis, ANSYS does not consider the loss of specimen, which is a real problem.
	\item Uneven density distribution of specimen or insufficient material purity. The specimen's purity will undoubtedly impact the physical properties and, thus, the experimental results.
	\item The load direction is not perpendicular to the specimen surface. This results in generating a partial force perpendicular to the specimen surface, which reduces the displacement under the same load.
	\item The friction between the support structure and the specimen surface can impact the experiment.\cite{ref1}
\end{enumerate}

So, there are some methods can optimise this experiment and get more accurate data.
\begin{enumerate}
	\item Multiple experiments to reduce the effect of data errors.
	\item Calibration of the machine before the start of the experiment to prevent Instrumental Error caused by the load not being perpendicular to the specimen surface and by the machine's initial values being inaccurate.
	\item Experimentation with new and higher purity parts and completion of three load tests using different specimens.
	\item Use a smoother support structure or a smoother specimen to reduce the effect of friction.\cite{ref1}
\end{enumerate}
\label{sec:conclusion}
