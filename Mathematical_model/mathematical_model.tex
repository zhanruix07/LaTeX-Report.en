\section{Section A}
\label{sec:problem_description}
\subsection{Results of the experiment}
 Through the experiments, we obtained data about the bending displacement and force of two materials, Aluminium and Mild Steel, respectively, and obtained Figure \ref{F 1.1}.
  
 \begin{figure}
 	\centering
 \subfloat[50N loading Mild Steel]{\includegraphics[width=9cm]{"Introduction/figures/Steel50"}}
 \subfloat[100N loading Mild Steel]{\includegraphics[width=9cm]{"Introduction/figures/Steel100"}}
 	\\
 	\centering
	\subfloat[150N loading Mild Steel]{\includegraphics[width=9cm]{"Introduction/figures/150Steel"}}
 	\subfloat[50N loading Aluminium]{\includegraphics[width=9cm]{"Introduction/figures/Al50"}}
 	\\
 		\centering
 	\subfloat[100N loading Aluminium]{\includegraphics[width=9cm]{"Introduction/figures/Al100"}}
 	\subfloat[150N loading Aluminium]{\includegraphics[width=9cm]{"Introduction/figures/Al150"}}
 	\caption{Results of experiments with Steel and Aluminium} %图片标题
 	\label{F 1.1}
 \end{figure}

 

The moment of inertia for the beam of cross-section can be calculated by equation \begin{large}
$I=\frac{b\times h^3}{12}$
\end{large} with the width and height of the experimental sample being constants. I is $4.5\times \textbf{}10^{-11}$. The results for the modulus of elasticity can be derived from the calculation of equation \begin{large}$E=\frac{PL^3}{48I\delta}$ \end{large}, which can be obtained in Table 1.2 .
\begin{figure}
	\centering
	\includegraphics[width=1.0\linewidth]{"Mathematical_model/figures/modulus of elasticity"}
	\caption*{}
	\label{T 1.2}
\end{figure}
\subsection{Discussion and summary}
The modulus of elasticity of Mild Steel and Aluminium is 172.6698GPa and 63.7500GPa respectively, which shows that the modulus of elasticity of Mild Steel is higher than that of Aluminium, proving that Mild Steel is more rigid and has less displacement under the same load.

The theoretical moduli of elasticity of Mild Steel and Aluminium are 200 GPa and 71 GPa respectively, but the experimental results are smaller.


\FloatBarrier
